\abstract{РЕФЕРАТ}

\setlength{\parindent}{1.25cm}

Объем работы равен \formbytotal{lastpage}{страниц}{е}{ам}{ам}. Работа содержит \formbytotal{figurecnt}{иллюстраци}{ю}{и}{й}, \formbytotal{tablecnt}{таблиц}{у}{ы}{}, \arabic{bibcount} библиографических источников и \formbytotal{числоПлакатов}{лист}{}{а}{ов} графического материала. Количество приложений – 2. Графический материал представлен в приложении А. Фрагменты исходного кода представлены в приложении Б.

Перечень ключевых слов: корпоративный мессенджер, система, сервер, web-приложение, группы, каналы, сообщения, администратор, пользователь база данных, пользователи, сообщения, группы, безопасность, шифрование, файлы, логирование, API, производительность, администрирование, API, аутентификация, авторизация, приложение, модуль.

Объектом разработки является корпоративный мессенджер, обеспечивающий безопасное взаимодействие сотрудников внутри организации, защиту паролей, управление доступом и интеграцию с внутренними системами компании.

Целью выпускной квалификационной работы является создание корпоративного коммуникационного инструмента, обеспечивающего обмен сообщениями, файлами и уведомлениями между сотрудниками, а также возможность администрирования групп, настройки прав и контроля доступа.

В процессе создания мессенджера были выделены основные сущности путем разработки архитектуры базы данных, использованы классы и методы модулей, обеспечивающие работу с сообщениями, пользователями и группами, а также безопасное шифрование данных. Разработаны разделы, содержащие возможности системы, структуру управления, интерфейс для пользователей.

При разработке системы использовалась собственная архитектура с использованием Python, Webob, SQLite и криптографических алгоритмов.

\selectlanguage{english}
\abstract{ABSTRACT}
  
The volume of work is \formbytotal{lastpage}{page}{}{s}{s}. The work contains \formbytotal{figurecnt}{illustration}{}{s}{s}, \formbytotal{tablecnt}{table}{}{s}{s}, \arabic{bibcount} bibliographic sources and \formbytotal{числоПлакатов}{sheet}{}{s}{s} of graphic material. The number of applications is 2. The graphic material is presented in annex A. The layout of the site, including the connection of components, is presented in annex B.

List of keywords: corporate messenger, system, server, web application, groups, channels, messages, administrator, user database, users, messages, groups, security, encryption, files, logging, API, performance, administration, API, authentication, authorization, application, module.

The object of development is a corporate messenger that ensures secure interaction of employees within the organization, password protection, access control and integration with the company's internal systems.

The goal of the final qualification work is to create a corporate communication tool that ensures the exchange of messages, files and notifications between employees, as well as the ability to administer groups, configure rights and access control.

In the process of creating the messenger, the main entities were identified by developing the database architecture, classes and methods of modules were used to work with messages, users and groups, as well as secure data encryption. Sections containing system capabilities, a management structure, and an interface for users were developed.

The system was developed using our own architecture, using Python, Webob, SQLite and cryptographic algorithms.

\selectlanguage{russian}
