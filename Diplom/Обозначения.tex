\section*{ОБОЗНАЧЕНИЯ И СОКРАЩЕНИЯ}

БД -- база данных.

ИС -- информационная система.

ИТ -- информационные технологии. 

КТС -- комплекс технических средств.

ОМТС -- отдел материально-технического снабжения. 

ПО -- программное обеспечение.

РП -- рабочий проект.

СУБД -- система управления базами данных.

ТЗ -- техническое задание.

ТП -- технический проект.

API (Application Programming Interface) -- интерфейс, через который клиент (браузер) обменивается JSON-данными с сервером по протоколу HTTP.

FK (Foreign Key) -- внешний ключ, который связывает данные между таблицами, ссылаясь на PK другой таблицы.

ORM (Object-Relational Mapping) -- технология, которая автоматически связывает объекты в коде с таблицами в базе данных.

PK (Primary Key) -- первичный ключ, уникальный идентификатор записи в таблице.

SQL (Structured Query Language) -- язык структурированных запросов для работы с базами данных (создание, чтение, изменение, удаление данных).

UML (Unified Modelling Language) -- язык графического описания для объектного моделирования в области разработки программного обеспечения.