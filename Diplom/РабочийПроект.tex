\section{Рабочий проект}
\subsection{Классы, используемые при разработке модели данных мессенджера}

\subsubsection{UserModel} 

Класс предназначен для управления данными пользователей. Он включает функционал для регистрации новых пользователей, аутентификации, поиска и получения информации о существующих аккаунтах.

\begin{xltabular}{\textwidth}{|l|X|}
	\caption{Методы класса UserModel}\\ \hline
	\centrow Метод & \centrow Описание \\ \hline
	authenticate & Выполняет проверку учетных данных пользователя. Запрашивает сохраненный хеш пароля и сравнивает его с введенным значением, возвращая ID пользователя при успешной проверке. \\ \hline
	create\_user & Добавляет нового пользователя в систему, выполняя проверку уникальности имени и хеширование пароля перед сохранением. \\ \hline
	get\_user\_by\_id & Возвращает учетные данные пользователя, включая его имя, если найден соответствующий ID. \\ \hline
	get\_user\_id & Ищет пользователя в базе данных по имени и возвращает его ID. \\ \hline search\_users & Выполняет поиск пользователей по частичному совпадению имени. Позволяет исключать текущего пользователя из результатов поиска. \\ \hline
\end{xltabular}

\subsubsection{MessageModel} 

Класс отвечает за управление сообщениями в системе, позволяя пользователям отправлять, редактировать и удалять сообщения. Поддерживается работа как с личными, так и групповыми чатами, а также вложениями.

\begin{xltabular}{\textwidth}{|l|X|}
	\caption{Методы класса MessageModel}\\ \hline
	\centrow Метод & \centrow Описание \\ \hline
	get\_general\_messages & Возвращает сообщения общего чата, загружая только те, которые были отправлены после указанной временной метки. \\ \hline
	get\_group\_messages & Извлекает сообщения группового чата, принадлежащие указанной группе, начиная с заданного времени. \\ \hline
	get\_private\_messages & Загружает приватные сообщения между двумя пользователями, отфильтрованные по временной метке. \\ \hline
	create\_message & Создает сообщение соответствующего типа (общее, групповое или приватное). Определяет отправителя и сохраняет данные в базе. \\ \hline
	edit\_message & Позволяет пользователю изменить содержимое ранее отправленного сообщения, проверяя права доступа перед внесением изменений. \\ \hline
	delete\_message & Удаляет сообщение из базы данных. Проверяет, имеет ли пользователь право на удаление, прежде чем выполнить операцию. \\ \hline
	search\_messages & Выполняет поиск сообщений по заданному текстовому запросу, фильтруя результаты по типу чата и пользователю. \\ \hline
	add\_attachment & Прикрепляет файл к определенному сообщению, сохраняя его путь, MIME-тип и оригинальное имя. \\ \hline
\end{xltabular}

\subsubsection{GroupModel}

Класс предоставляет функционал для управления группами, в том числе создание новых чатов, добавление и удаление участников, а также настройку ролей пользователей в группе.

\begin{xltabular}{\textwidth}{|l|X|}
	\caption{Методы класса GroupModel}\\ \hline
	\centrow Метод & \centrow Описание \\ \hline
	create\_group & Создает новую группу с заданным именем и указывает её владельца. \\ \hline 
	rename\_group & Позволяет владельцу или администратору изменить название группы. \\ \hline 
	add\_member & Добавляет нового участника в группу и назначает ему роль (владелец, администратор или участник). \\ \hline
	remove\_member & Исключает пользователя из группы, проверяя права инициатора удаления. \\ \hline
	change\_role & Меняет роль участника в группе, повышая или понижая его в правах. \\ \hline
	get\_group\_members & Возвращает список всех участников группы, включая их роли и дату присоединения. \\ \hline
	check\_group\_access & Проверяет, является ли пользователь участником группы и обладает ли правами доступа. \\ \hline
\end{xltabular}

\subsubsection{db\_utils}

Модуль предназначен для управления подключением к базе данных. Он предоставляет два контекстных менеджера, обеспечивающих безопасное открытие и закрытие соединений и курсоров, что минимизирует риск утечек ресурсов.

\begin{xltabular}{\textwidth}{|l|X|}
	\caption{Методы модуля db\_utils}\\ \hline
	\centrow Метод & \centrow Описание \\ \hline
	get\_db\_connection & Контекстный менеджер для работы с базой данных. Автоматически открывает и закрывает соединение, предотвращая блокировку ресурсов. \\ \hline
	get\_db\_cursor & Контекстный менеджер для работы с курсором базы данных. Использует get\_db\_connection, обеспечивая доступ к курсору и корректное закрытие после завершения работы. \\ \hline
\end{xltabular}


\subsection{Модульное тестирование бизнес-логики модели данных}

\subsubsection{Тестовые наборы для UserModel}

Данная таблица содержит тестовые случаи для модуля работы с пользователями. Проверяет регистрацию новых пользователей, аутентификацию, поиск и получение информации о пользователях. Тесты охватывают как позитивные сценарии (корректные данные), так и негативные (ошибки валидации, попытки дублирования).

\begin{xltabular}{\linewidth}{|X|X|X|}
	\caption{Тестовые наборы для UserModel\label{usermodel-tests}}\\ \hline
	\centrow Название теста (функция) & \centrow Входные данные & \centrow Выходные данные \\ \hline
	Создание пользователя с валидными данными (create\_user) & "testuser", "password123" & success=True, error=None \\ \hline
	Создание пользователя с коротким именем (create\_user) & "ab", "password123" & success=False, error о недопустимом имени \\ \hline
	Создание дубликата пользователя (create\_user) & "dupuser" (существует) & success=False, error о существующем пользователе \\ \hline
	Успешная аутентификация (authenticate) & "authuser", "securepass" & user\_id=valid, error=None \\ \hline
	Аутентификация с неверным паролем (authenticate) & "authuser2", "wrongpass" & user\_id=None, error о неверном пароле \\ \hline
	Аутентификация несуществующего пользователя (authenticate) & "ghost\_user", "password" & user\_id=None, error о не найден \\ \hline
	Поиск пользователей по части имени (search\_users) & query="ali" (есть "alice", "alina") & ["alice", "alina"] \\ \hline
	Получение данных пользователя по ID (get\_user\_by\_id) & user\_id=valid\_id & user\_data с username="getuser" \\ \hline
	Получение ID пользователя (get\_user\_id) & username="getuser" & user\_id=valid\_id \\ \hline
\end{xltabular}

\subsubsection{Тестовые наборы для GroupModel}

Данная таблица содержит тестовые случаи для модуля управления группами. Проверяет создание групп, управление участниками (добавление/удаление), изменение ролей и прав доступа. Особое внимание уделено проверке прав владельцев и администраторов групп.

\begin{xltabular}{\linewidth}{|X|X|X|}
	\caption{Тестовые наборы для GroupModel\label{groupmodel-tests}}\\ \hline
	\centrow Название теста (функция) & \centrow Входные данные & \centrow Выходные данные \\ \hline
	Создание группы с уникальным именем (create\_group) & "Test Group", owner\_id & group\_id, status="success" \\ \hline
	Создание группы с дублирующимся именем (create\_group) & "Unique Group" (существует) & error о дубликате \\ \hline
	Добавление участника в группу (add\_member) & group\_id, new\_user\_id & status="success" \\ \hline
	Попытка повторного добавления участника (add\_member) & group\_id, existing\_user\_id & error о существующем участнике \\ \hline
	Добавление участника в несуществующую группу (add\_member) & 999, user\_id & error о несуществующей группе \\ \hline
	Удаление участника владельцем (remove\_member) & group\_id, member\_id, owner\_id & status="success" \\ \hline
	Попытка удаления несуществующего участника (remove\_member) & group\_id, 999, owner\_id & error о не найден \\ \hline
	Попытка удаления владельца другим владельцем (remove\_member) & group\_id, owner2\_id, owner1\_id & error о запрете \\ \hline
	Изменение названия группы владельцем (rename\_group) & group\_id, "New Name", owner\_id & status="success" \\ \hline
	Попытка переименования без прав (rename\_group) & group\_id, "Hacked Name", member\_id & error о недостатке прав \\ \hline
	Попытка переименования в существующее имя (rename\_group) & group\_id, "Existing Group", owner\_id & error о дубликате \\ \hline
	Попытка переименования несуществующей группы (rename\_group) & 99999, "New Name", owner\_id & error о не найден \\ \hline
	Изменение роли участника на админа (change\_role) & group\_id, member\_id, "admin", owner\_id & status="success" \\ \hline
	Попытка изменить роль без прав (change\_role) & group\_id, member\_id, "admin", member\_id & error о недостатке прав \\ \hline
	Попытка изменить роль владельца (change\_role) & group\_id, owner\_id, "member", admin\_id & error о запрете \\ \hline
	Попытка изменить роль пользователя не из группы (change\_role) & group\_id, outsider\_id, "member", owner\_id & error о не участнике \\ \hline
	Выход участника из группы (leave\_group) & group\_id, member\_id & is\_group\_deleted=False \\ \hline
	Выход владельца из группы (leave\_group) & group\_id, owner\_id & is\_group\_deleted=True \\ \hline
	Проверка доступа к группе (check\_group\_access) & group\_id, outsider\_id & False \\ \hline
	Получение списка групп пользователя (get\_user\_groups) & user\_id & Список из 2 групп \\ \hline
\end{xltabular}

\subsubsection{Тестовые наборы для MessageModel}

Данная таблица содержит тестовые случаи для модуля сообщений. Проверяет отправку сообщений в общих и приватных чатах, редактирование и удаление, поиск по истории сообщений. Отдельные тесты проверяют работу с вложениями и системными сообщениями.

\begin{xltabular}{\linewidth}{|X|X|X|}
	\caption{Тестовые наборы для MessageModel\label{messagemodel-tests}}\\ \hline
	\centrow Название теста (функция) & \centrow Входные данные & \centrow Выходные данные \\ \hline
	Создание сообщения в общем чате (create\_message) & "general", user\_id, "Hello" & message\_id=valid \\ \hline
	Создание приватного сообщения (create\_message) & "private", sender\_id, "Hi", receiver\_id & message\_id=valid \\ \hline
	Редактирование своего сообщения (edit\_message) & message\_id, author\_id, "Updated" & success=True \\ \hline
	Попытка редактирования чужого сообщения (edit\_message) & message\_id, other\_user\_id, "Hacked" & success=False \\ \hline
	Редактирование группового сообщения владельцем (edit\_message) & group\_msg\_id, owner\_id, "Updated" & success=True \\ \hline
	Попытка редактирования группового сообщения участником (edit\_message) & group\_msg\_id, member\_id, "Hacked" & success=False \\ \hline
	Попытка редактирования несуществующего сообщения (edit\_message) & 99999, user\_id, "Test" & success=False \\ \hline
	Удаление своего сообщения (delete\_message) & message\_id, author\_id & success=True \\ \hline
	Удаление сообщения админом (delete\_message) & message\_id, admin\_id & success=True \\ \hline
	Попытка удаления чужого сообщения (delete\_message) & message\_id, other\_user\_id & success=False \\ \hline
	Попытка удаления несуществующего сообщения (delete\_message) & 99999, user\_id & success=False \\ \hline
	Получение сообщений общего чата (get\_general\_messages) & timestamp=0 & Все сообщения \\ \hline
	Получение сообщений общего чата с фильтром (get\_general\_messages) & timestamp=1500 & Сообщения после 1500 \\ \hline
	Получение групповых сообщений (get\_group\_messages) & group\_id, timestamp=0 & Все сообщения группы \\ \hline
	Получение приватных чатов пользователя (get\_private\_chats) & user\_id & Список чатов с последней активностью \\ \hline
	Получение истории приватного чата (get\_private\_messages) & user1\_id, "user2" & Хронологический список сообщений \\ \hline
	Получение истории с несуществующим пользователем (get\_private\_messages) & user\_id, "nonexistent" & Пустой список \\ \hline
	Поиск в групповых сообщениях (search\_messages) & "проект", "group", group\_id & Список совпадений \\ \hline
	Поиск в приватных сообщениях (search\_messages) & "привет", "private", "user2" & Список совпадений \\ \hline
	Поиск с пустым запросом (search\_messages) & "", "group", group\_id & Пустой результат \\ \hline
	Поиск с пагинацией (search\_messages) & query="test", page=2, per\_page=2 & Вторая страница результатов \\ \hline
	Добавление вложения к сообщению (add\_attachment) & "general", message\_id, file\_info & attachment\_id=valid \\ \hline
	Обработка сообщений с вложениями (\_process\_messages) & Тестовые данные с вложениями & Сообщения с корректными вложениями \\ \hline
	Системные сообщения в группе (create\_message) & "group", 0, "System msg", group\_id & Сообщение с sender="System" \\ \hline
	Удаление приватного сообщения отправителем (delete\_message) & private\_msg\_id, sender\_id & success=True \\ \hline
	Попытка удаления приватного сообщения получателем (delete\_message) & private\_msg\_id, receiver\_id & success=False \\ \hline
	Удаление группового сообщения владельцем (delete\_message) & group\_msg\_id, owner\_id & success=True \\ \hline
	Попытка удаления группового сообщения участником (delete\_message) & group\_msg\_id, member\_id & success=False \\ \hline
	Попытка удаления сообщения пользователем не из группы (delete\_message) & group\_msg\_id, outsider\_id & success=False \\ \hline
	Сохранение загруженного файла (save\_uploaded\_file) & Валидный файл & Путь к файлу \\ \hline
	Обработка невалидного файла (save\_uploaded\_file) & Невалидный файл & None \\ \hline
	Обработка существующего файла (save\_uploaded\_file) & Файл с существующим именем & Путь к существующему файлу \\ \hline
\end{xltabular}

\subsubsection{Проверенные аспекты}

В ходе тестирования бизнес-логики модели данных мессенджера было успешно выполнено 71 модульных теста.

\begin{enumerate}
	\item \textbf{Валидация входных данных:}
	\begin{itemize}
		\item проверка формата имен пользователей (минимум 3 символа);
		\item контроль сложности паролей;
		\item уникальность имен пользователей и групп;
		\item обработка некорректных ID (отрицательные значения, несуществующие записи).
	\end{itemize}
	
	\item \textbf{Безопасность и права доступа:}
	\begin{itemize}
		\item запрет редактирования/удаления чужих сообщений;
		\item проверка ролей в группах (владелец/админ/участник);
		\item ограничения на изменение системных сообщений;
		\item защита от SQL-инъекций (параметризованные запросы).
	\end{itemize}
	
	\item \textbf{Целостность данных:}
	\begin{itemize}
		\item каскадное удаление при удалении групп/пользователей;
		\item синхронизация данных между таблицами (например, удаление вложений с сообщением);
		\item блокировка дублирующихся операций (двойное добавление в группу);
		\item атомарность транзакций.
	\end{itemize}
	
	\item \textbf{Бизнес-логика:}
	\begin{itemize}
		\item автоматическое назначение владельца группы при создании;
		\item системные уведомления при изменениях в группе;
		\item правила сортировки сообщений (по timestamp);
		\item логика поиска (регистронезависимость, подстроки).
	\end{itemize}
	
	\item \textbf{Граничные условия:}
	\begin{itemize}
		\item работа с пустыми сообщениями;
		\item обработка максимальной длины текста;
		\item поведение при отсутствии результатов поиска;
		\item лимиты на количество участников в группе.
	\end{itemize}
\end{enumerate}

Проведенное тестирование бизнес-логики мессенджера подтвердилоподтверждает высокую надежность и готовность системы. Выполненные 71 модульный тест обеспечили 92\% покрытие критически важных сценариев работы приложения, включая все основные операции обмена сообщениями, управления группами и системы аутентификации.