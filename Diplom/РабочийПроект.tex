\section{Рабочий проект}
\subsection{Классы, используемые при разработке модели данных мессенджера}

\subsubsection{UserModel} 

Класс предназначен для управления данными пользователей. Он включает функционал для регистрации новых пользователей, аутентификации, поиска и получения информации о существующих аккаунтах.

\begin{xltabular}{\textwidth}{|l|X|}
	\caption{Методы класса UserModel}\\ \hline
	\centrow Метод & \centrow Описание \\ \hline
	authenticate & Выполняет проверку учетных данных пользователя. Запрашивает сохраненный хеш пароля и сравнивает его с введенным значением, возвращая ID пользователя при успешной проверке. \\ \hline
	create\_user & Добавляет нового пользователя в систему, выполняя проверку уникальности имени и хеширование пароля перед сохранением. \\ \hline
	get\_user\_by\_id & Возвращает учетные данные пользователя, включая его имя, если найден соответствующий ID. \\ \hline
	get\_user\_id & Ищет пользователя в базе данных по имени и возвращает его ID. \\ \hline search\_users & Выполняет поиск пользователей по частичному совпадению имени. Позволяет исключать текущего пользователя из результатов поиска. \\ \hline
\end{xltabular}

\subsubsection{MessageModel} 

Класс отвечает за управление сообщениями в системе, позволяя пользователям отправлять, редактировать и удалять сообщения. Поддерживается работа как с личными, так и групповыми чатами, а также вложениями.

\begin{xltabular}{\textwidth}{|l|X|}
	\caption{Методы класса MessageModel}\\ \hline
	\centrow Метод & \centrow Описание \\ \hline
	get\_general\_messages & Возвращает сообщения общего чата, загружая только те, которые были отправлены после указанной временной метки. \\ \hline
	get\_group\_messages & Извлекает сообщения группового чата, принадлежащие указанной группе, начиная с заданного времени. \\ \hline
	get\_private\_messages & Загружает приватные сообщения между двумя пользователями, отфильтрованные по временной метке. \\ \hline
	create\_message & Создает сообщение соответствующего типа (общее, групповое или приватное). Определяет отправителя и сохраняет данные в базе. \\ \hline
	edit\_message & Позволяет пользователю изменить содержимое ранее отправленного сообщения, проверяя права доступа перед внесением изменений. \\ \hline
	delete\_message & Удаляет сообщение из базы данных. Проверяет, имеет ли пользователь право на удаление, прежде чем выполнить операцию. \\ \hline
	search\_messages & Выполняет поиск сообщений по заданному текстовому запросу, фильтруя результаты по типу чата и пользователю. \\ \hline
	add\_attachment & Прикрепляет файл к определенному сообщению, сохраняя его путь, MIME-тип и оригинальное имя. \\ \hline
\end{xltabular}

\subsubsection{GroupModel}

Класс предоставляет функционал для управления группами, в том числе создание новых чатов, добавление и удаление участников, а также настройку ролей пользователей в группе.

\begin{xltabular}{\textwidth}{|l|X|}
	\caption{Методы класса GroupModel}\\ \hline
	\centrow Метод & \centrow Описание \\ \hline
	create\_group & Создает новую группу с заданным именем и указывает её владельца. \\ \hline 
	\_group & Позволяет владельцу или администратору изменить название группы. \\ \hline 
	\_member & Добавляет нового участника в группу и назначает ему роль (владелец, администратор или участник). \\ \hline
	remove\_member & Исключает пользователя из группы, проверяя права инициатора удаления. \\ \hline
	change\_role & Меняет роль участника в группе, например, повышая до администратора. \\ \hline
	get\_group\_members & Возвращает список всех участников группы, включая их роли и дату присоединения. \\ \hline
	check\_group\_access & Проверяет, является ли пользователь участником группы и обладает ли правами доступа. \\ \hline
\end{xltabular}

\subsubsection{get\_db\_cursor}

Контекстный менеджер, предназначенный для управления подключением к базе данных. Позволяет автоматически открывать и закрывать курсор, предотвращая утечки соединений.

\begin{xltabular}{\textwidth}{|l|X|}
	\caption{Методы класса get\_db\_cursor}\\ \hline
	\centrow Метод & \centrow Описание \\ \hline
	enter & Открывает соединение с базой данных и создает объект курсора. \\ \hline
	exit & Закрывает курсор и соединение при выходе из контекста. \\ \hline
\end{xltabular}

\subsection{Модульное тестирование разработанного web-сайта}

Модульный тест для класса User из модели данных представлен на рисунке \ref{unitUser:image}.

\begin{figure}[ht]
\begin{lstlisting}[language=Python]
from django.test import TestCase
from .models import *
User = get_user_model()


class ShpoTestCases(TestCase):

    def setUp(self) -> None:
        self.user = User.objects.create(username='testtestovich', password='testtestovich', first_name='Sad', last_name='')

    def test_2(self):

        self.assertEqual(self.user.first_name, 'Sad')
        self.assertEqual(self.user.last_name, 'Cat')
        print((self.user))
        print((self.user.first_name))
        print((self.user.last_name))
\end{lstlisting}  
\caption{Модульный тест класса User}
\label{unitUser:image}
\end{figure}

\subsection{Системное тестирование разработанного web-сайта}

На рисунке \ref{main:image} представлена главная страница сайта «Русатом – Аддитивные технологии».
\newpage % при необходимости можно переносить рисунок на новую страницу
\begin{figure}[H] % H - рисунок обязательно здесь, или переносится, оставляя пустоту
\center{\includegraphics[width=1\linewidth]{main1}}
\center{\includegraphics[width=1\linewidth]{main2}}
\center{\includegraphics[width=1\linewidth]{main3}}
\caption{Главная страница сайта «Русатом – Аддитивные технологии»}
\label{main:image}
\end{figure}

На рисунке \ref{menu:image} представлен динамический вывод заголовков, включающий в себя искомые фразы при поиске фраз.

\begin{figure}[ht]
\center{\includegraphics[width=1\linewidth]{menu}}
\caption{Динамический вывод заголовков}
\label{menu:image}
\end{figure}

На рисунке \ref{enter:image} представлен ввод данных для публикации новости.

\begin{figure}[ht]
\center{\includegraphics[width=1\linewidth]{enter}}
\caption{Ввод данных для публикации очень-очень длинной, интересной и полезной новости}
\label{enter:image}
\end{figure}
