\section*{ВВЕДЕНИЕ}
\addcontentsline{toc}{section}{ВВЕДЕНИЕ}

В условиях растущего давления санкций и ограничений на использование зарубежного программного обеспечения российским компаниям становится все сложнее поддерживать эффективные внутренние коммуникации. Многие популярные корпоративные мессенджеры либо вводят ограничения для пользователей из России, либо полностью прекращают работу на территории страны. В такой ситуации критически важно иметь независимое решение, не зависящее от зарубежных поставщиков и доступное для использования без риска отключения или потери данных.

Закрытые корпоративные мессенджеры отечественной разработки позволяют компаниям обходить санкционные барьеры, обеспечивая стабильную работу внутренних коммуникаций. Они функционируют без зависимости от западных облачных сервисов, не требуют сторонних API и имеют полное управление данными внутри организации. Такие решения позволяют сохранить бизнес-процессы, поддерживать работу команд и исключить риски внезапных отключений.

Ранее корпоративные сети и коммуникационные платформы в России во многом зависели от западных технологий, что делало их уязвимыми перед санкционными угрозами. Развитие отечественного программного обеспечения позволяет создать гибкие и надежные системы, которые легко адаптируются под нужды компании. Современные российские корпоративные мессенджеры интегрируются с внутренними сервисами предприятия, обеспечивают удобное управление пользователями, защиту данных и полную автономность от зарубежных сервисов.

Главной задачей независимого корпоративного мессенджера является предоставление стабильной платформы для общения, управления бизнес-процессами и обмена файлами без ограничений со стороны иностранных разработчиков.

Цель настоящей работы – разработка корпоративного мессенджера, полностью независимого от санкций и обеспечивающего бесперебойную внутреннюю коммуникацию. Для достижения поставленных целей необходимо решить следующие задачи: – провести анализ предметной области и текущих санкционных ограничений; – разработать концептуальную модель приложения, учитывающую автономность системы; – спроектировать архитектуру корпоративного мессенджера без зависимости от иностранных серверов; – реализовать систему на языке программирования Python с локальным управлением данными.