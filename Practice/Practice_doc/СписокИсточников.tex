\addcontentsline{toc}{section}{СПИСОК ИСПОЛЬЗОВАННЫХ ИСТОЧНИКОВ}

\begin{thebibliography}{9}
	
	\bibitem{abramov} Абрамов, С. М. Python: создание веб-сайтов / С. М. Абрамов. – Москва~: Эксмо, 2020. – 320 с. – ISBN 978-5-04-109098-2. – Текст~: непосредственный.
	
	\bibitem{brown} Браун, Э. Python и анализ данных / Э. Браун. – Санкт-Петербург~: БХВ-Петербург, 2019. – 400 с. – ISBN 978-5-9775-4105-3. – Текст~: непосредственный.
	
	\bibitem{vasiliev} Васильев, А. Н. Python на примерах. Практический курс по программированию / А. Н. Васильев. – Санкт-Петербург~: Наука и Техника, 2017. – 432 с. – ISBN 978-5-94387-982-6. – Текст~: непосредственный.
	
	\bibitem{golub} Голубь, Н. Python. Основы программирования / Н. Голубь. – Санкт-Петербург~: Питер, 2021. – 416 с. – ISBN 978-5-4461-1431-9. – Текст~: непосредственный.
	
	\bibitem{zlatopolsky} Златопольский, Д. М. Основы программирования на языке Python / Д. М. Златопольский. – Москва~: ДМК Пресс, 2021. – 288 с. – ISBN 978-5-97060-873-1. – Текст~: непосредственный.
	
	\bibitem{lukin} Лукин, С. Н. Разработка веб-приложений на Python с использованием фреймворка Flask / С. Н. Лукин. – Москва~: БИНОМ. Лаборатория знаний, 2018. – 288 с. – ISBN 978-5-9963-3580-9. – Текст~: непосредственный.
	
	\bibitem{mckinley} Маккинли, У. Python и обработка данных / У. Маккинли. – Москва~: ДМК Пресс, 2016. – 482 с. – ISBN 978-5-97060-392-7. – Текст~: непосредственный.
	
	\bibitem{muller} Мюллер, Дж. Python для чайников / Дж. Мюллер. – Москва~: Диалектика, 2019. – 416 с. – ISBN 978-5-907114-25-3. – Текст~: непосредственный.
	
	\bibitem{ramalho} Рамальо, Л. Python. К вершинам мастерства / Л. Рамальо. – Москва~: ДМК Пресс, 2016. – 768 с. – ISBN 978-5-97060-384-2. – Текст~: непосредственный.
	
	\bibitem{reitz} Рейтц, К. Разработка Web API на Python / К. Рейтц, Т. Шахэм. – Санкт-Петербург~: Питер, 2014. – 272 с. – ISBN 978-5-496-01153-2. – Текст~: непосредственный.
	
	\bibitem{sedlacek} Седлачек, Р. Django 3. Практика создания современных Web-приложений / Р. Седлачек. – Санкт-Петербург~: БХВ-Петербург, 2021. – 416 с. – ISBN 978-5-9775-0661-8. – Текст~: непосредственный.
	
	\bibitem{simonov} Симонов, М. Python. Самое необходимое / М. Симонов. – Санкт-Петербург~: БХВ-Петербург, 2020. – 320 с. – ISBN 978-5-9775-0647-2. – Текст~: непосредственный.
	
	\bibitem{stevenson} Стивенсон, Т. Python. Самоучитель / Т. Стивенсон. – Москва~: ДМК Пресс, 2017. – 322 с. – ISBN 978-5-97060-510-5. – Текст~: непосредственный.
	
	\bibitem{fedorov} Федоров, Д. Ю. Программирование на языке Python / Д. Ю. Федоров. – Москва~: БИНОМ. Лаборатория знаний, 2019. – 224 с. – ISBN 978-5-9963-3581-6. – Текст~: непосредственный.
	
	\bibitem{chernyshev} Чернышев, А. Python. Исчерпывающее руководство / А. Чернышев. – Санкт-Петербург~: БХВ-Петербург, 2018. – 512 с. – ISBN 978-5-9775-3898-5. – Текст~: непосредственный.
	
\end{thebibliography}