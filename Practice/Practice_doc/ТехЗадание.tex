\section{Техническое задание}
\subsection{Основание для разработки}

Основанием для разработки является задание на выпускную квалификационную работу бакалавра "<Бизнес-проект: Закрытый корпоративный мессенджер">.

\subsection{Цель и назначение разработки}

Разработка защищенного корпоративного мессенджера для внутреннего обмена сообщениями сотрудников компании с функциями:
\begin{itemize}
	\item шифрование передаваемых сообщений;
	\item групповое взаимодействие с ролевым доступом;
	\item хранение истории переписки.
\end{itemize}

	
\subsection{Функционал мессенджера}

\subsubsection{Аутентификация}
Система аутентификации обеспечивает безопасный доступ к корпоративному мессенджеру через следующие механизмы:

Регистрация новых пользователей:

	\begin{itemize}
		\item проверка уникальности корпоративного логина;
		\item валидация сложности пароля (минимум 8 символов, обязательное использование цифр и специальных символов);
		\item шифрование паролей с использованием алгоритма bcrypt.
	\end{itemize}

\subsubsection{Чаты и сообщения}
Система обмена сообщениями предоставляет следующие возможности:

Личная переписка:

	\begin{itemize}
		\item шифрование для конфиденциальных сообщений;
		\item поиск по истории личных сообщений;
		\item отправка вложений разного типа.
	\end{itemize}
	
Групповые чаты:

	\begin{itemize}
		\item создание тематических каналов;
		\item прикрепление файлов до 100 МБ.
	\end{itemize}
	
Поиск по сообщениям:

	\begin{itemize}
		\item фильтрация по дате;
		\item отображение найденных сообщений.
	\end{itemize}
	
\subsubsection{Управление группами}

Система управления группами включает:
	
Управление участниками:

	\begin{itemize}
		\item приглашение в чат;
		\item повышение/понижение в ролях;
		\item полное исключение из группы.
	\end{itemize}
	
Система ролей:

	\begin{itemize}
		\item владелец (полные права управления);
		\item администратор (ограниченные права модерации);
		\item участник (базовые права общения);
		\item гость (ограниченный доступ только для чтения).
	\end{itemize}
	

\subsection{Роли пользователей}

Система ролей в корпоративном мессенджере реализована по трехуровневой иерархической модели, обеспечивающей дифференцированный контроль доступа и управления чатами. Основу системы составляют три ключевые роли: владелец, администратор и участник, каждая из которых наделена строго определённым набором полномочий.

Владелец представляет высший уровень привилегий, обладая исключительными правами на управление структурой групп. В его компетенцию входит создание и удаление чатов, назначение администраторов, модификация названий групп, а также полный контроль над составом участников. Владелец имеет неограниченный доступ к сообщениям и может удалять любой контент в пределах своей группы.

Администратор занимает промежуточное положение в иерархии, получая от владельца делегированные полномочия для повседневного управления чатом. В круг его обязанностей входит пополнение состава участников, регулирование прав обычных пользователей, модерация сообщений и мониторинг активности. При этом администратор не может изменять статус владельца или вносить структурные изменения в группу.

Участник обладает базовым набором функций, необходимым для полноценного общения. Ему доступны отправка сообщений, создание личных чатов, просмотр истории переписки и добровольный выход из групп. Участники не имеют прав на управление составом или содержанием групповых чатов.

Архитектурно система реализована с соблюдением принципа независимости – каждая группа поддерживает собственный набор ролей, не влияющий на права пользователей в других чатах. Назначение ролей происходит по строгой иерархии: владелец назначает администраторов, которые в свою очередь управляют обычными участниками. Создатель группы автоматически получает статус владельца, что исключает ситуации бесхозных чатов.

Система ролей применяется исключительно к групповым чатам – личная переписка строится на принципах равноправия всех участников. Такой подход обеспечивает баланс между гибкостью управления корпоративными коммуникациями и защитой от злоупотреблений административными полномочиями. Реализованная модель соответствует принципу минимальных привилегий, предоставляя каждому пользователю ровно тот уровень доступа, который необходим для выполнения его функций.

\begin{xltabular}{\textwidth}{|X|X|}
	\caption{Роли пользователей}\label{tab:roles} \\ \hline
	\centrow Роль & \centrow Права \\ \hline
	\endfirsthead
	Обычный пользователь ("Участник") & 
	\begin{itemize}
		\item личная переписка;
		\item участие в групповых чатах;
		\item отправка сообщений и файлов.
	\end{itemize} \\ \hline
	Администратор чата ("Админ") & 
	\begin{itemize}
		\item все права обычного пользователя;
		\item добавление/удаление участников;
		\item переименование чата.
	\end{itemize} \\ \hline
	Создатель чата ("Владелец") &
	\begin{itemize}
		\item все права администратора чата;
		\item возможность удалять группу.
	\end{itemize} \\ \hline
\end{xltabular}

\subsubsection{Тестирование и отладка}  
Процесс тестирования и отладки включает:  
\begin{itemize}  
	\item проверка безопасности;  
	\item функциональное тестирование интерфейсов;  
	\item тестирование удобства использования.  
\end{itemize}  
\newpage
\subsection{Требования к интерфейсу}

\subsubsection{Основные элементы интерфейса}

Главная страница мессенджера разделена на три основные колонки, обеспечивающие удобную навигацию и взаимодействие.

\begin{figure}[ht]
	\centering
	\includegraphics[width=0.8\linewidth]{"images/UI макет"}
	\caption{Схема интерфейса мессенджера}
	\label{fig:ui-main}
\end{figure}

В левой части интерфейса расположена панель, позволяющая пользователю свободно перемещаться по групповым чатам и личным сообщения. Список участников отображается с указанием времени их последней активности в формате часов и минут, что позволяет быстро оценить, кто из пользователей недавно был онлайн.

Центральная (и по совместительству основная) колонка предназначена для отображения сообщений из выбранного чата, с возможностью в нижней части колонки отправлять новые сообщения с помощью поля для ввода текста. Также предусмотрена возможность поиска сообщений по ключевым фразам, словам и даже буквам.

Правая колонка содержит информацию об участниках текущего чата и их ролях. Эта колонка позволяет быстро переключаться между личными чатами с участниками и управлять их ролями через контекстное меню.

Страница регистрации аккаунта представляет собой форму, содержащую все необходимые элементы для создания нового профиля пользователя. 

\begin{figure}[ht]
	\centering
	\includegraphics[width=0.8\linewidth]{"images/UI макет регистрации"}
	\caption{Схема интерфейса формы регистрации}
	\label{fig:ui-reg}
\end{figure}

Основное внимание сосредоточено на трёх ключевых элементах формы. Поле "Никнейм" предназначено для ввода уникального имени пользователя, которое будет отображаться в системе. Ниже расположено поле "Пароль", где можно указать секретную комбинацию для защиты аккаунта. Завершающим элементом формы является кнопка "Зарегистрироваться", которая подтверждает введённые данные и создаёт новый аккаунт. В нижней части страницы размещена вспомогательная ссылка, позволяющая существующим пользователям быстро перейти на страницу авторизации.

Центральное место на странице авторизации занимает форма входа. Форма состоит из двух основных полей ввода: "Никнейм" для указания имени пользователя и "Пароль" для ввода секретной комбинации, обеспечивающей безопасность аккаунта. Под полями ввода расположена кнопка "Войти", которая инициирует процесс аутентификации после заполнения необходимых данных. В нижней части страницы предусмотрена вспомогательная ссылка, позволяющая новым пользователям мгновенно перейти на страницу регистрации, если у них ещё не создан профиль.

\begin{figure}[ht]
	\centering
	\includegraphics[width=0.8\linewidth]{"images/UI макет авторизации"}
	\caption{Схема интерфейса формы авторизации}
	\label{fig:ui-auth}
\end{figure}

\subsection{Сценарии использования}

\subsubsection{Регистрация нового пользователя}  
Процедура регистрации нового пользователя включает следующие шаги:  
\begin{itemize}  
	\item пользователь нажимает кнопку "Зарегистрироваться" на форме авторизации. 
\end{itemize}

Система отображает форму регистрации (рис. \ref{fig:ui-reg}) с полями:  

\begin{itemize}  
	\item имя пользователя ("никнейм");  
	\item пароль (с требованиями сложности);  
	\item подтверждение пароля.
\end{itemize}



\begin{itemize}
	\item пользователь заполняет все обязательные поля;  
	\item пользователь нажимает кнопку "Зарегистрироваться";  
	\item система проверяет данные.
\end{itemize}

При корректных данных: 

\begin{itemize}  
	\item создаёт новую учётную запись;  
	\item отправляет подтверждение на корпоративную почту;  
	\item перенаправляет на форму авторизации;  
	\item выводит сообщение "Регистрация успешно завершена".
\end{itemize}

В случаях обнаружения системой ошибок:

\begin{itemize}
	\item выделяет проблемные поля.
\end{itemize}

Система показывает соответствующие сообщения об ошибках:  

\begin{itemize}
	\item "Пароль должен содержать не менее 8 символов";  
	\item "Пароли не совпадают";  
	\item "Учётная запись с таким именем уже существует".   
\end{itemize}

После успешной регистрации администратор получает уведомление о новом пользователе для подтверждения корпоративного доступа. 

\subsubsection{Авторизация пользователя}  
Процесс авторизации выполняется по схеме:  
\begin{itemize}  
	\item пользователь открывает веб-интерфейс мессенджера;  
	\item система отображает форму авторизации (рис. \ref{fig:ui-auth});  
	\item пользователь вводит корпоративный логин и пароль;  
	\item пользователь нажимает кнопку "Войти";  
	\item система проверяет учётные данные;
		\item при успехе — загружает основной интерфейс (рис. \ref{fig:ui-main});  
		\item при ошибке — показывает сообщение "Неверный логин или пароль";   
	\item при нажатии "Зарегистрироваться" система перенаправляет на форму регистрации (рис. \ref{fig:ui-reg}).  
\end{itemize}  

\subsubsection{Создание группового чата}  
Алгоритм создания группового чата:  
\begin{itemize}  
	\item пользователь нажимает кнопку "+" (Создать чат) на левой панели;
\end{itemize}

После чего система отображает диалоговое окно: 

\begin{itemize}
	\item поле ввода названия чата;  
	\item список доступных сотрудников;  
	\item чекбоксы для выбора участников.
\end{itemize}

Далее:

\begin{itemize}		
	\item пользователь вводит название чата;  
	\item пользователь отмечает нужных участников;  
	\item пользователь нажимает кнопку "Создать".
\end{itemize}

Система:

\begin{itemize}		
\item создаёт новый чат;  
\item добавляет выбранных участников;  
\item отображает новый чат в списке.  
\end{itemize} 

\subsubsection{Отправка сообщений}  
Логика отправки сообщений:  
\begin{itemize}  
	\item пользователь выбирает чат из списка;  
	\item система загружает историю переписки;  
	\item пользователь вводит текст в нижнее поле ввода.
\end{itemize}

Пользователь может:

\begin{itemize}
		\item нажать кнопку "Отправить" (или Enter);  
		\item нажать кнопку "Прикрепить файл" и выбрать файл.
\end{itemize}

После чего:

\begin{itemize}
		\item шифрует и отправляет сообщение;  
		\item отображает сообщение в истории чата;  
		\item для файлов — показывает превью и название.   
\end{itemize}  

\subsubsection{Управление участниками группы (для администраторов)}  
Процедура управления участниками:  
\begin{itemize}  
	\item пользователь открывает групповой чат;  
	\item пользователь нажимает иконку "Управление чатом" в заголовке.  
\end{itemize}

Система отображает меню:

\begin{itemize}  
	\item "Добавить участника";  
	\item "Исключить участника";  
	\item "Назначить администратора".  
\end{itemize}

При выборе "Добавить участника":  

\begin{itemize}
	\item открывается список сотрудников;  
	\item администратор выбирает сотрудников;  
	\item нажимает "Добавить";  
	\item система присылает уведомление новым участникам.
\end{itemize}

При выборе "Исключить участника":  

\begin{itemize}
	\item открывается список текущих участников;  
	\item администратор выбирает участника;  
	\item нажимает "Исключить";  
	\item система удаляет участника из чата.  
\end{itemize}

\subsubsection{Поиск сообщений в чате}
Процедура поиска сообщений в чате включает следующие шаги:

\begin{itemize}
	\item пользователь нажимает кнопку "Поиск" в интерфейсе чата (рис. \ref{fig:ui-main});
\end{itemize}

Система отображает модальное окно поиска с элементами:
\begin{itemize}
	\item поле ввода поискового запроса;
	\item выпадающий список для выбора типа чата (общий/групповой/личный);
	\item параметры сортировки (по дате/по релевантности);
	\item кнопка "Найти".
\end{itemize}

\begin{itemize}
	\item пользователь вводит поисковый запрос в текстовое поле;
	\item при необходимости выбирает тип чата для поиска;
	\item устанавливает параметры сортировки;
	\item нажимает кнопку "Найти".
\end{itemize}

При успешном выполнении поиска:
\begin{itemize}
	\item система отображает список найденных сообщений с пагинацией;
	\item каждое сообщение показывает:
	\begin{itemize}
		\item текст сообщения с подсветкой совпадений;
		\item имя отправителя;
		\item дату и время отправки;
		\item название чата/группы (для групповых сообщений).
	\end{itemize}
	\item предоставляет возможность перейти к сообщению в контексте чата.
\end{itemize}

В случаях возникновения ошибок:
\begin{itemize}
	\item при отсутствии результатов показывает сообщение "Сообщения не найдены";
	\item при ошибке соединения с базой данных выводит сообщение "Ошибка поиска, попробуйте позже".
\end{itemize}

Система логирует все поисковые запросы для анализа активности пользователей и улучшения поискового алгоритма.

\subsection{Требования к оформлению документации}

Документация должна соответствовать ГОСТ 19.102-77 и ГОСТ 34.601-90. Единая система программной документации.